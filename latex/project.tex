\documentclass[12pt] {article}

\usepackage[margin=1in]{geometry} %one inch margins
\renewcommand{\baselinestretch}{2} %double space, safe for fancy headers
\usepackage{pslatex} %Times font
\usepackage{apacite} %apa citation style
\bibliographystyle{apacite}
%\usepackage[pdfborder={0 0 0}]{hyperref}%for hyperlinks without ugly boxes
\usepackage{graphicx} %for figures
\usepackage{enumerate} %for lists
\usepackage{fancyhdr} %header
\pagestyle{fancy}
\usepackage[font={small,sf},format=plain,labelfont=bf,up]{caption}
\usepackage{indentfirst}

\fancyhf{}
\fancyhead[l,lo]{Qiao Zhang \textit{ CSE8317 Project Report}} %left top header
\fancyhead[r,ro]{\thepage} %right top header

\begin{document}
\title{Building a Knowledge Graph from Historical Defect Reports with Orthogonal Defect Classification}
\author{Qiao Zhang}
\date \today
\maketitle

\thispagestyle{empty}

\bigskip
%\tableofcontents
\pagebreak
\setcounter{page}{1}
\section{Introduction}
Defect report is commonly a structured natural language artifact generated during the software reliability engineering process.
A well-structured defect report would be very informative to viewers and very efficient for knowledge acquisition purpose.
The previous experience (or knowledge) of handling a specific type of defect would be critical to the success of the future defect analysis.
However, retrieving knowledge from the historical defect reports could be a painful job especially when the volume of reports is high enough.
Based on the experience of a well-known telecommunication equipment company in China, the number of their historical defect reports has reached 3 million and the number of newly generated defect reports is about 30 thousand each month.
Without an appropriate approach to standardize, formalize, and organize the historical defect reports, it is basically impossible to retrieve knowledge from the gigantic pool of reports.\par

With orthogonal defect classification (ODC), we can not only classify the defect reports into a few specific types but also fit newly generated defect reports into some pre-determined workflows to improve the engineering efficiency.
Ontology, as a formal and explicit specification of a shared conceptualization with a complex semantic web structure, is able to standardize and formalize a domain knowledge or the expert interpretation \cite{christina2016an}.



\pagebreak
\bibliography{../bib/mybib} %rename to your .bib file
\end{document}
